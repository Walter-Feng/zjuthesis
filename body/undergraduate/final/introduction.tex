\cleardoublestylepage{common}

\section{绪论}


20世纪发展出的量子力学彻底改变了我们对于化学学科的认知:电子在稳定分子中总是以本征态的形式存在的性质以及费米子的性质给化学观带来了史无前例的改变。我们开始讨论分子轨道,开始考虑电子分布对分子间作用力的影响,开始逐渐明白我们曾经所讨论的化学键的真正的物理含义。利用量子力学第一性原理发展出来的Hartree-Fock (HF)方法,密度泛函理论(Density Functional Theory, DFT)等量子化学方法让我们能够清晰地看到一个分子具有的电子结构,能够利用理论判断不同构型之间的能量大小;而简谐振子模型和线性响应理论的引入使光谱的计算真正变为了可能,我们能够更为直接地将理论计算与实验结果相关联,并部分做到理论指导实践。

量子力学的核心——薛定谔方程(Schr\"odinger's equation)作为线性偏微分方程,其线性代数框架允许人们使用计算机进行模拟,而其完备性又允许人们使用非常庞大的基组使结果非常接近实验值。事实上,现有的完全组态关联方法(Full Configuration Interaction, Full-CI)已经为电子结构领域中的精确解。 同样地,而在量子动力学领域中有较为精确的离散变量表象(Discrete Variable Representation,DVR),其通过一个基于空间的基组来展开初始波函数,同时将体系的哈密顿算符转化为矩阵,从而进行含时演化。其优点包括完整考虑薛定谔方程带来的量子效应,同时由于采用固定的空间基组,其哈密顿矩阵以及对应的演化矩阵是固定的,从而在实际计算过程中能够利用线性代数加速库进行加速,获得可观的运行效率。

然而为了保证有足够数量的空间基组对初始波函数以及势能函数进行模拟,在使用DVR方法时需要引入数量可观的基组,模拟正常化学反映体系时的基组数量将按照体系自由度指数级上升,这对于一般计算模拟是无法承受的,因此我们需要利用经典近似来优化,舍弃不必要的高阶量子修正项,降低复杂度,将最重要的量子效应纳入模拟体系,而这便是混合经典量子动力学或半经典量子动力学的初衷。这样的方法对于模拟诸如原子核等量子效应有限但仍旧重要的系统自由度的模拟尤其重要。

