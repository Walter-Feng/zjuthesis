\section{问题提出的背景}

\par 正文格式与具体要求\cite{zjuthesisrules}

\subsection{背景介绍}
化学反应的本质在于电子的再分配,而化学问题的核心在于原子核和电子的相对运动。反应机理体现了历史以来化学家对于化学反应的一个直觉的判断的积累:化学反应总是正电荷吸引负电荷,电子总是从富电子的领域流入缺电子的领域。然而由于电子运动的复杂性,所有反应机理以及反应原理只能停留在纯经验阶段:不存在一个完整的数理逻辑链,总是不可避免地带有``例外''。而量子化学的发展,则彻底改变了这样的局面。

\subsubsection{电子结构}
20世纪发展出的量子力学彻底改变了我们对于化学学科的认知:电子在稳定分子中总是以本征态的形式存在的性质以及费米子的性质给化学观带来了史无前例的改变。我们开始讨论分子轨道,开始考虑电子分布对分子间作用力的影响,开始逐渐明白我们曾经所讨论的化学键的真正的物理含义。利用量子力学第一性原理发展出来的Hartree-Fock (HF)方法,密度泛函理论(Density Functional Theory, DFT)等量子化学方法让我们能够清晰地看到一个分子具有的电子结构,能够利用理论判断不同构型之间的能量大小;而简谐振子模型和线性响应理论的引入使光谱的计算真正变为了可能,我们能够更为直接地将理论计算与实验结果相关联,并部分做到理论指导实践。

量子力学的核心——薛定谔方程(Schr\"odinger's equation)作为线性偏微分方程,由于基组完备性以及线性代数框架

\subsubsection{动力学}

然而现有的电子结构理论都建立在一个最基本的近似:波恩-奥本海默近似,即原子核固定不动。因此电子结构理论并不能预测化学反应,不能研究化学反应过程中的电子运动以及原子核运动的情况。而最为广泛应用的分子动力学理论(Molecular Dynamics, MD)只运用了最经典的牛顿力学框架,即分子运动只服从其余分子产生的分子力场,且直接省略了电子结构的描述。在这样的框架下分化学反应的模拟基本无法实现,而我们较为关心的一些量子效应,包括电子跃迁,配位成键等过程皆无法予以精确的描述。


\subsubsection{项目提出的原因}

\section{本研究的意义和目的}
