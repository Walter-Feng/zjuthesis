\section{问题提出的背景}
化学反应的本质在于电子的再分配,而化学问题的核心在于原子核和电子的相对运动。反应机理体现了历史以来化学家对于化学反应的一个直觉的判断的积累:化学反应总是正电荷吸引负电荷,电子总是从富电子的领域流入缺电子的领域。然而由于电子运动的复杂性,所有反应机理以及反应原理只能停留在纯经验阶段:不存在一个完整的数理逻辑链,总是不可避免地带有``例外''。而量子化学的发展,则彻底改变了这样的局面。

\subsection{背景介绍}

\subsubsection{电子结构}
20世纪发展出的量子力学彻底改变了我们对于化学学科的认知:电子在稳定分子中总是以本征态的形式存在的性质以及费米子的性质给化学观带来了史无前例的改变。我们开始讨论分子轨道,开始考虑电子分布对分子间作用力的影响,开始逐渐明白我们曾经所讨论的化学键的真正的物理含义。利用量子力学第一性原理发展出来的Hartree-Fock (HF)方法,密度泛函理论(Density Functional Theory, DFT)等量子化学方法让我们能够清晰地看到一个分子具有的电子结构,能够利用理论判断不同构型之间的能量大小;而简谐振子模型和线性响应理论的引入使光谱的计算真正变为了可能,我们能够更为直接地将理论计算与实验结果相关联,并部分做到理论指导实践。

量子力学的核心——薛定谔方程(Schr\"odinger's equation)作为线性偏微分方程,其线性代数框架允许人们使用计算机进行模拟,而其完备性又允许人们使用非常庞大的基组使结果非常接近实验值。事实上,现有的完全组态关联方法(Full Configuration Interaction, Full-CI)已经为电子结构领域中的精确解。然而使用庞大的基组意味着更大的内存需要以及更长的运行时间,而其$O(N^6)$的复杂度使该方法无法应用于现实分子体系。对于包括蛋白质、DNA等大分子,或是复杂晶体的模拟,即使是最传统的Hartree-Fock方法或是密度泛函理论也很难进行模拟。因此如何使用近似方法进行模拟为电子结构理论的一大热门,其中紧束缚方法(Tight Binding, TB)应用最广。

\subsubsection{动力学}

然而现有的电子结构理论都建立在一个最基本的近似:波恩-奥本海默近似,即原子核固定不动。因此电子结构理论并不能预测化学反应,不能研究化学反应过程中的电子运动以及原子核运动的情况。而最为广泛应用的分子动力学理论(Molecular Dynamics, MD)只运用了最经典的牛顿力学框架,即分子运动只服从其余分子产生的分子力场,且直接省略了电子结构的描述。在这样的框架下化学反应的模拟基本无法实现,而我们较为关心的一些量子效应,包括电子跃迁,配位成键等过程皆无法予以精确的描述。因此我们需要使用合适的含时模拟框架。

在完备基组下,薛定谔方程的含时演化可通过本征态的相位改变来完成,而这也是一般的线性偏微分方程常用的解析求解含时演化的方式\cite{courant2008methods}:

\begin{quote}
若$\{H_k'\}$和$\{|H_k'\rangle\}$分别为以$\hat{H}$为哈密顿量的系统下的本征值和本征向量,即
\begin{equation}
	\hat{H} |H'_k\rangle = H_k' |H'_k\rangle
\end{equation}
且在初始条件下有
\begin{equation}
	|\psi \rangle\big|_{t=0} = \sum_k c_k |H_k'\rangle
\end{equation}
那么
\begin{equation}
	|\psi \rangle\big|_{t} = \sum_k c_k e^{-iH_k t}|H_k'\rangle
\end{equation}
\end{quote}

然而由于具体模型过于复杂,我们无法解析给出该体系的所有独立的本征态,从而不可避免地出现一个有限基组的截断过程。而这样的截断使得运用同样的方法进行含时演化会造成巨大的误差。此时我们需要一个另外的基于空间的基组来展开初始波函数,同时将体系的哈密顿算符转化为矩阵,从而进行含时演化,而在量子动力学领域中享有``量子精确解''称号的离散变量表象(Discrete Variable Representation,DVR)方法则基本属于这样的思想\cite{colbert1992novel}。其优点包括完整考虑薛定谔方程带来的量子效应,同时由于采用固定的空间基组,其哈密顿矩阵以及对应的演化矩阵是固定的,从而在实际计算过程中能够利用线性代数加速库进行加速,获得可观的运行效率。

然而为了保证有足够数量的空间基组对初始波函数以及势能函数进行模拟,在使用DVR方法时需要引入数量可观的基组,如在王林军老师的文章\cite{LinjunSemiclassical}的一维摩尔斯势振动模型中使用了$[-0.76,2.00]$间隔为0.01的格点,共276个基组。而实际体系几乎无法化简为只有一个维度的模型,需要多个维度同时模拟,此时基组数量将以指数级上升,这对于一般计算模拟是无法承受的,因此我们需要利用经典近似来优化,舍弃不必要的高阶量子修正项,将最重要的量子效应纳入模拟体系,而这便是混合经典量子动力学或半经典量子动力学的初衷。这样的方法对于模拟诸如原子核等量子效应有限但仍旧重要的系统自由度的模拟尤其重要。

\subsection{项目提出的原因}
王林军老师提出的半经典磨雅动力学即为一种半经典量子动力学方法\cite{LinjunSemiclassical}。该方法从Oleg Prezhdo的量子哈密顿动力学方法\cite{OlegQHD}(Quantum Hamiltonian Dynamics, QHD)中获取了灵感,但使用相空间表象规避了在海森堡表象下动力学方程推导繁琐的问题。在相空间表象下,坐标算符与动量算符都能够以其各自的变量的形式给出而无需满足海森堡的共轭变量的对易关系\cite{wigner1997quantum},从而避免了坐标算符与动量算符的交换不对称性问题,使任意阶演化方程都能够按照磨雅提出的相空间对易子\cite{moyal1949quantum}给出。然而在该方法中某一阶的变量由于哈密顿量的存在会依赖于更高阶的变量,从而需要通过辅助相空间分布的形式近似求出该高阶变量,从而完成方程截断,形成完整的迭代模拟过程。然而文章使用的辅助相空间分布为系统各变量平均值为中心的多项式高斯函数,该方法存在两大缺点:
\begin{itemize}
	\item 高斯函数的局域性过强,对于距中心较远,靠近边界的领域的描述不足;
	\item 无法描述多中心相空间分布。
\end{itemize}
其中的第二点最为致命,其直接导致了该模型无法完成退相干效应的模拟,也无法完成诸如量子反射、量子衍射等在量子力学模拟中同样重要的现象,使其很难运用于如电子跃迁、电荷转移等过程的模拟。同时使用辅助相空间分布会导致在较大时间尺度上的数值不稳定性\cite{kaiguPSQHD}。

解决上述问题的最好方法即为建立多中心基组\cite{kaiguPSQHD},通过多轨迹的方式克服单中心基组存在的这样两个缺点。而顾锴的研究与Martens的研究都同时指明了对初始波函数进行Wigner变换后只使用经典分子动力学方法进行模拟也能得到与离散变量表象方法相当接近的结果\cite{kaiguPSQHD,donoso2000simulation}。
\section{本研究的意义和目的}
