\cleardoublestylepage{common}

\section{绪论}


20世纪发展出的量子力学彻底改变了我们对于化学学科的认知:电子在稳定分子中总是以本征态的形式存在的性质以及费米子的性质给化学观带来了史无前例的改变。我们开始讨论分子轨道,开始考虑电子分布对分子间作用力的影响,开始逐渐明白我们曾经所讨论的化学键的真正的物理含义。利用量子力学第一性原理发展出来的Hartree-Fock (HF)方法,密度泛函理论(Density Functional Theory, DFT)等量子化学方法让我们能够清晰地看到一个分子具有的电子结构,能够利用理论判断不同构型之间的能量大小;而简谐振子模型和线性响应理论的引入使光谱的计算真正变为了可能,我们能够更为直接地将理论计算与实验结果相关联,并部分做到理论指导实践。

在完备基组下,薛定谔方程的含时演化可通过本征态的相位改变来完成,而这也是一般的线性偏微分方程常用的解析求解含时演化的方式\cite{courant2008methods}。然而由于具体模型过于复杂,我们无法解析给出该体系的所有独立的本征态,从而不可避免地出现一个有限基组的截断过程。而这样的截断使得运用同样的方法进行含时演化会造成误差。此时我们需要一个另外的基于空间的基组来展开初始波函数,同时将体系的哈密顿算符转化为矩阵,从而进行含时演化,而在量子动力学领域中较为精确的的离散变量表象(Discrete Variable Representation,DVR)方法则基本属于这样的思想\cite{colbert1992novel}。其优点包括完整考虑薛定谔方程带来的量子效应,同时由于采用固定的空间基组,其哈密顿矩阵以及对应的演化矩阵是固定的,从而在实际计算过程中能够利用线性代数加速库进行加速,获得可观的运行效率。

然而为了保证有足够数量的空间基组对初始波函数以及势能函数进行模拟,在使用DVR方法时需要引入数量可观的基组,模拟正常化学反映体系时的基组数量将按照体系自由度指数级上升,这对于一般计算模拟是无法承受的,因此我们需要利用经典近似来优化,舍弃不必要的高阶量子修正项,降低复杂度,将最重要的量子效应纳入模拟体系,而这便是混合经典量子动力学或半经典量子动力学的初衷。这样的方法对于模拟诸如原子核等量子效应有限但仍旧重要的系统自由度的模拟尤其重要。

量子哈密顿动力学(Quantum Hamiltonian Dynamics, QHD)最先开发出了以期望值为核心的半经典量子动力学的演化方法\cite{OlegQHD}。其采用一系列位置算符与动量算符的幂来描述一个系统在某一个时刻的状态,而非通过计算化学方法中常用的波函数来描述体系。在QHD体系中任意算符$A$的期望值的演化通过Ehrenfest定理来进行:
\begin{equation}
	\frac{d}{d t}\langle\hat{A}\rangle=\left\langle\frac{\partial}{\partial t} \hat{A}\right\rangle+\left\langle\frac{1}{i \hbar}[\hat{A}, \hat{H}]\right\rangle
\end{equation}
其中$[\cdot, \cdot]$为算符对易子,
\begin{equation}
	[\hat{A}, \hat{B}] \equiv \hat{A}\hat{B} - \hat{B} \hat{A}
\end{equation}
对于一阶项算符,即位置算符$X_j$和对应动量算符$P_j$,Ehrenfest定理直接对应了经典力学中的牛顿定律,
\begin{equation}
\begin{aligned}
&\frac{d}{d t}\left\langle X_{j}\right\rangle_{\psi(t)}=\frac{1}{m_{j}}\left\langle P_{j}\right\rangle_{\psi(t)}\\
&\frac{d}{d t}\left\langle P_{j}\right\rangle_{\psi(t)}=\left\langle-\frac{\partial V}{\partial x_{j}}\right\rangle_{\psi(t)}
\end{aligned}
\end{equation}
这在狄拉克看来是由于在量子力学中的对易子即代表了经典力学中的泊松括号,在接近经典极限($\hbar \rightarrow 0$)时对易子必须要转化成泊松括号\cite{dirac1981principles}。因此1阶QHD(QHD-1)与经典力学严格对应,而高阶项能够对经典力学演化加入量子力学修正,而若使用无穷数量的高阶项进行修正,则能够得到量子精确解,这正如同在概率论中通过无穷数量的中心矩估计能够推出整体的概率分布\cite{fisz2018probability}。

而QHD方法的最主要问题在于动力学方程的推导的繁琐。由于整个演化过程采用希尔波特空间,位置算符与动量算符之间必须要满足不对易关系:
\begin{equation}
	\left[X_{r}, P_{s}\right]= i\hbar \delta_{r s}
\end{equation}
因此算符的乘积需要考虑算符的顺序,同时动量算符需表达为关于位置的偏导算符,
\begin{equation}
	P_j = - i \hbar \frac{\partial}{\partial X_j}
\end{equation}
从而给高阶动力学方程的推导造成了极大的麻烦。也正由于这样的原因,在现有的QHD以及其延伸方法中不存在高于五阶的计算。

王林军老师和沈一帆在原有的量子哈密顿动力学的基础上发展出了相空间量子哈密顿动力学,即半经典磨雅动力学\cite{YifanShenPSQHD,LinjunSemiclassical}。该方法充分运用了Wigner发展出的相空间表象下的波函数表达方法,其最突出的特点在于相空间下的动量、位置算符的乘积与线性组合可直接等效于薛定谔方程下的算符写法而无需将转换成动量算符转化为关于位置算符的偏导的形式\cite{wigner1997quantum},从而位置算符与动量算符对易,消除了算符序对于期望值的影响。Wigner变换能够实现波函数从实空间到相空间的转换,而相空间下的动力学演化方程由磨雅(Moyal)提出的磨雅括号给出\cite{moyal1949quantum}。然而由于哈密顿量的存在,某一个由位置算符和动量算符的乘积组成的力学量$q^i p^j$会依赖于更高阶的量$q^k p^l$,即$k+l > i+j$.为得到更高阶的力学量,原文章采用了以系统位置,动量的期望值为中心的多项式高斯函数来拟合系统的相空间分布。

另一方面,顾锴的工作与Martens的工作都指出了经典维格纳近似(Classical Wigner Approximation, CWA)方法进行相空间分布演化的出色的稳定性,而结果与量子精确解相比十分可观\cite{kaiguPSQHD,donoso2000simulation}。这首先依赖于最初的波函数体现出的量子效应——相较于普通的粒子,波函数存在着一定的展宽,从而能够对全空间有一个扫描的作用。同时从磨雅括号的形式可以看出经典力学项——泊松括号在演化中能够占主要部分,而在高阶量的参与由于$\left(\frac{\hbar}{2}\right)^{2j}$的放缩因子的存在而影响较小,从而能够得到与量子精确解相近的结果。

经典维格纳近似的大部分优点来源于狄拉克函数只使用其所代表的粒子的相空间坐标,但其缺陷也来源于此——由于位置和动量的演化只需服从经典力学,因此量子力学修正很难在经典维格纳近似中体现出来。同时由于在经典维格纳近似中每个狄拉克函数的配比不变,在整个演化过程中相空间一直处于非负的状态,而这与量子精确解得到的非稳态相空间分布可局部为负数相违背\cite{heller1976wigner},从而可能导致进一步的误差,同时经典维格纳近似给出的相空间分布可能并不适合提供高阶量的期望值,从而使经典维格纳近似与半经典磨雅动力学结合带来一定的困难。另一方面,虽然有较多工作致力于改进半经典维格纳近似的精度\cite{heller1976wigner,liu2007real},但他们并不具备一个可以随意更改精度的框架,无法做到提供足够的计算量使最终结果与量子精确解无限接近的方法,从而在一定程度上局限了这些方法的应用。在这一点上半经典磨雅动力学能够具备任意调整精度与可无限接近量子精确解这两大特性,然而其最大的缺点在于其较差的稳定性\cite{YifanShenPSQHD,kaiguPSQHD}。高斯函数的单中心性质使量子力学中一些重要的性质,如退相干效应,量子反射等,不容易得到描述,存在极大的误差,最终发生数值崩坏。

通过对建立在经典维格纳近似方法的SMD框架进行系统研究,我们希望能够解决SMD方法的最基本的问题,即数值稳定性问题。为保证数值稳定性,我们可能需要充分发挥狄拉克函数在相空间优化中存在的优势,探讨尽可能的优化方法,使整体的力学量的误差达到最小,使演化结果能够进一步接近量子精确解。


