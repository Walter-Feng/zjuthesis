{
    \setlength{\parindent}{0em}
    \vspace{0.5em}
    \par {\zihao{3}\bfseries 一、题目:\hfill \Title \hfill} \\
    \par {\zihao{3}\bfseries 二、指导教师对文献综述、开题报告、外文翻译的具体要求}
}

文献综述需要系统介绍磨雅动力学方法的研究进展,以及SMD作为一种量子修正方法的背景、意义和进展,重点从原理、实现方法和效果上进行全面系统的探讨,要求逻辑清晰、行文流畅,文献引用全面。

开题报告需在文献总数基础上,进一步总结已有方法的特点和不足,凝练科学问题,提出解决方案。结合该领域的发展趋势和本课题目标,对进一步研究进行初步的设计,并为其中可能遇到的问题做好应对措施。要求研究方法具有创新性,研究方案翔实可行,确立有限目标。
 \vfill

\signature{指导教师(签名)}
