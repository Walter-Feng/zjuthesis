\cleardoublepage
\chapter{文献综述}

\section{背景介绍}
薛定谔方程(Schr\"odinger's equation)的含时演化一直都是量子化学中一个重要的分支。在完备基组下,薛定谔方程的含时演化可通过本征态的相位改变来完成,而这也是一般的线性偏微分方程常用的解析求解含时演化的方式\cite{courant2008methods}:

\begin{quote}
若$\{H_k'\}$和$\{|H_k'\rangle\}$分别为以$\hat{H}$为哈密顿量的系统下的本征值和本征向量,即
\begin{equation}
	\hat{H} |H'_k\rangle = H_k' |H'_k\rangle
\end{equation}
且在初始条件下有
\begin{equation}
	|\psi \rangle\big|_{t=0} = \sum_k c_k |H_k'\rangle
\end{equation}
那么
\begin{equation}
	|\psi \rangle\big|_{t} = \sum_k c_k e^{-iH_k t}|H_k'\rangle
\end{equation}
\end{quote}

然而由于具体模型过于复杂,比如一个分子的电子结构,我们无法解析给出该体系的所有独立的本征态,从而不可避免地出现一个有限基组的截断过程。而这样的截断使得运用同样的方法进行含时演化会造成巨大的误差。此时我们需要一个另外的基于空间的基组来展开初始波函数,同时将体系的哈密顿算符转化为矩阵,从而进行含时演化,而在量子动力学领域中享有``量子精确解''称号的离散变量表象(Discrete Variable Representation,DVR)方法则基本属于这样的思想\cite{colbert1992novel}。

\section{国内外研究现状}

\subsection{研究方向及进展}

\subsection{存在问题}

\section{研究展望}

\newpage
\printbibliography
