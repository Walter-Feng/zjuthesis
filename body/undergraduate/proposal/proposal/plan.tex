\section{研究计划进度安排及预期目标}

\subsection{进度安排}
\begin{itemize}
	\item 2020年2月:熟悉\emph{entos}的程序框架的同时训练C++14的能力,同时完成多维度的基于粒子化Wigner变换的半经典磨雅动力学的关键公式的推导;
	\item 2020年3月:嵌入多维度离散变量方法与基于粒子化Wigner变换的半经典磨雅动力学的主要框架,与\emph{entos}的接口进行匹配;
	\item 2020年4月:完成程序,对热门的一维模型进行测试,同时反向评估程序运行情况,对可能的流程优化进行探讨;
	\item 2020年5月:对各种可能的优化方法进行测试;若能够保证数值稳定性与结果的可靠性,将对多维模型进行测试。
\end{itemize}


\subsection{预期目标}
通过建立SMD框架下的粒子化相空间分布演化,我们预期能够获得相比于原有的依赖于辅助相空间分布的SMD框架更加良好的数值稳定性,同时能够拥有相比于普通的分子动力学模拟框架与量子精确解更加接近的演化结果。我们希望在此基础上探讨可能的流程优化的方法并研究多维系统的可能性,增加SMD方法的实际应用价值。